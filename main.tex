\documentclass[letterpaper, 10 pt, conference]{ieeeconf}  % Comment this line out
                                                          % if you need a4paper
%\documentclass[a4paper, 10pt, conference]{ieeeconf}      % Use this line for a4
                                                          % paper

\IEEEoverridecommandlockouts                              % This command is only
                                                          % needed if you want to
                                                          % use the \thanks command
\overrideIEEEmargins
% See the \addtolength command later in the file to balance the column lengths
% on the last page of the document

\usepackage[utf8]{inputenc}
\usepackage[T1]{fontenc}
\usepackage{graphicx}
\usepackage{float}
\overrideIEEEmargins

\title{Smart Sprinkler System with User Configuration and Real Time Feedback}
\author{Patrick Armstrong, Ryan Williamson, Caleb Edwards}
\date{February 2020}

\begin{document}

\maketitle

\begin{abstract}
Smart sprinkler irrigation control designed with Python and I2C using Raspberry Pi's, sensors, relay board, ESP32 micro-controller and a mobile/desktop interface that allows a user to schedule watering with real time feedback from moisture, weather and infrared sensors. Minimizing excess water usage with automated adjustments to reschedule watering zones based on real-time data.

\end{abstract}

\section{Introduction and Motivation}
This project will be a smart sprinkler controller based off of different weather sensors placed throughout a user's yard. A zone is a subsystem in a sprinkler system that is controlled from a valve box. There are multiple zones controlled by one valve box. The goal of this project is to provide real time feedback to the user to make informed decisions on when and for how long to water each zone. Our main objective coming into this project was to create something that was feasible. Having a project that pertains to an industrial/commercial industry will help us familiarize ourselves with skills that take an existing product/service and improve upon it by adjusting or re-implementing new methods while maintaining a cheap budget.

As of now, a person with a sprinkler controller has to go to their sprinkler box in their garage and manually enter times and duration for each zone in their yard. Not many people are sure how long or what time of day they should even water their yard. It is almost an educated guess for most people with controllers. Based off of these times and duration settings that are manually entered, the old controller will turn on and water that zone for the specified duration. We would like to make the controller smarter. 

Our system will be based on recommendations. There will be sensors that collect data throughout the day and get stored in our controller. Based on these results, a predictive algorithm that we will create will send a recommendation to the UI stating a duration and time that the user should water that zone. The user will then be able to accept these recommendations, stick with the default settings, or enter some manual settings based on their preference. 

Our controller will collect data from a hub station that will represent a weather station. This will allow the user to also have that extra functionality on their system, viewed through the UI. So if a user would like to base their input to the controller based on these results, they can do so. Each zone of the sprinkler system will have different sensors, including a moisture sensor placed in the dirt of that zone. The data fed to the controller from these results are how the recommendations are determined for unique times of each zone. The layout of how data will be transferred can be seen in Fig. 1.

\begin{figure*}
	\centering
	\begin{minipage}{1\textwidth}
		\centering
		\includegraphics[width=\textwidth]{Diagram.png}
		\caption{Block Diagram of Sprinkler Controller}
		\label{label1}
	\end{minipage}
\end{figure*}


\section{Project Tasks and Specific Task Interfaces}
There will be many different tasks involved in the creation of a smart sprinkler system. Below we will specify the different tasks needed for this project, including their difficulty and if the necessity of having them.

\subsection{Sensor Arrays}

\subsection{Communication Between ESP and Sensors}

\subsection{Communication Between ESP and Raspberry PI}

\subsection{User Interface}

\subsection{Database and Data Storage}

\subsection{Predictive Algorithm}

\subsection{Relay to Turn on Zones}

\section{Testing and Integration Strategy}
\subsection{Individual Testing}
Testing will be done with distinct and unique portions of the project. Every portion of the project that can be ran without integration will be tested separately, before integrating. Data being read from each sensor to an ESP will be tested separate from the rest of the project until we know we can integrate. The communication from the ESP to the PI will be done separate with small test data. This will allow us to know we get the communication correct, before we move on to integration. 

There will be testing on turning on the valve using a Pi. This portion of testing will be to make sure that a Pi with a relay can turn on the valve for a sprinkler system. 

\subsection{Integration Testing}
Once we have data from the sensors being read by the ESP, and valid communication from the Pi and the ESP; we will be able to integrate these tasks into data being read from the ESP and transferred to the Pi. This will then allow for testing on the predictive algorithm. Once we have data onto the Pi, we can test the algorithm to see if we have results that make sense from the data that we are actually reading. 

\section{Group Management and Communication Plan}
Our group communication plan consists of two different platforms. We will use Discord for real time, fast communication. We will use Github for the management of all tasks and document storage. 

Discord will be for all of our immediate communication. If we have a time pressing question that needs to be discussed, Discord allows real time communication. Discord also allows file sharing if the team needs to view something in that exact moment. The link to this application is found in [1].

Github will allow us to all have access to our repository, as well as view the project plan and tasks associated with this repo. The project portion of Github allows you to use a Kanban style management for the agile project management. This process will allow us to view the progress of each other to see if we are falling behind in different areas. This will allow us to manage our time and help others with certain tasks if we realize we are falling behind. The link to Github is found in [2]. 

During the semester we plan on meeting at least once a week to discuss how things are progressing. If one member is having trouble with a task, this time can be used to meet in person to resolve the problem. The time at which we meet each week will be flexible based on everyone's schedules the upcoming semester.

\section{Schedule and Milestones}
\subsection{Spring}
\begin{itemize}
    \item Finish initial design/research. - Team
    \item Finish and submit project proposal. - Team
    \item Start interface development. - Ryan
\end{itemize}

\subsection{Summer}
\begin{itemize}
    \item Connect sensors to Pi for testing - Team
    \item Collect data using chosen sensors - Team
    \item ESP testing - Caleb
\end{itemize}

\subsection{August}
\begin{itemize}
    \item Interface finished and testing - Ryan
    \item Get relay working with Pi to activate valves - Patrick
    \item Create algorithms per sensor - Team
\end{itemize}

\subsection{September}
\begin{itemize}
    \item Finalize sensors to use - Team
    \item Finalize Database layout - Team
    \item Build sensor arrays - Ryan and Patrick
    \item Integrate all parts with wired communications - Team
    \item Start developing wireless communication - Caleb
\end{itemize}

\subsection{October}
\begin{itemize}
    \item Combine algorithms together - Team
    \item Build sprinkler demo - Team
    \item Want wireless to be finished - Caleb
\end{itemize}

\subsection{November}
\begin{itemize}
    \item Wireless fully integrated - Caleb
    \item Demo tested - Team
    \item Touch-up/bug testing - Team
    \item Finish Documentation - Team
\end{itemize}

\subsection{December}
\begin{itemize}
    \item Present Project - Team
\end{itemize}

\section{Risk Assessment}
\subsection{Sensor Arrays}
Med Risk: We will be testing and collecting data for temperature, infrared and moisture sensors over the summer. Once we finalize our choice on sensors, we will build a sensor array for collecting and processing the signals. Ideally, all our sensors will operate on I2C and we should be able to efficiently grab data. Our real risk, lays in how effective and accurate our sensors produce data.

\subsection{Communication Between ESP and Sensors}
High Risk : The risk for this task is high because there can be multiple sensors trying to communicate with one ESP microcontroller. To mitigate this risk, one sensor will be added at a time. This will allow for steady progress when trying to add sensors as we move through the project and get to our final destination. Sensors that communicate with I2C will be sought after, since this is what ESP uses. This will lower the risk as we will not have to use different protocols with different sensors.

\subsection{Communication Between ESP and Raspberry PI}
High Risk : The risk for communication between the ESP and sensors is high. This is high due to the choice of using wireless communication. This will be done either with BLE or Wifi. To lessen the risk of this task, wired communication over I2C will be used. This will create a more stable and reliable communication that we are already used to. 

\subsection{User Interface}
Med/Low Risk: The interface is based upon an open source irrigation control system that is being actively updated and used. This interface however doesn't support our sensors and I will have to add it to the interface, thus the only true area of risk for this is whether or not we can add new sensor information.

\subsection{Database and Data Storage}
Med/Low Risk: Will use a raspberry pi with MySQL and python to create and manage our database. This will be used to hold user data from sensors.However, MySQL and python are very simplistic to work with and shouldn't give any troubles.

\subsection{Predictive Algorithm}
Low Risk: The algorithm is low risk due to the fact that if we cannot find verified or supported information on the data that we receive, then we can create our own custom algorithm. The algorithm will work no matter what, but it might not be as efficient as we would want it to be. This is what makes it low risk. The algorithm will work, but efficiency could be limited.

\subsection{Relay to Turn on Zones}
Low/Med Risk: We are using a Raspberry Pi 8-Relay Card add-on which is supported with our open source interface and has low risk given its universal usage and correlation with the Raspberry PI Model 3 A+.

\section{Bill of Materials}
\subsection{Sensors}
\begin{itemize}
  \item BME280/680 (Temp/Humidity/Pressure)
  \item DHT22 (Temp/Humidity)
  \item BH1750 (Light)
\end{itemize}

\subsection{Communications/Compute/Electronic Boards}
\begin{itemize}
  \item 1-2x Raspberry Pi 3 B+
  \item 1x Raspberry Pi Zero W
  \item 4-5x ESP32
  \item (Relay Board)
\end{itemize}

\subsection{Sprinkler Equipment}
\begin{itemize}
  \item 4x (Sprinkler Valves)
\end{itemize}

\section{Vendor List}

\section{Conclusion}

\bibliographystyle{IEEEtran}
\bibliography{biblio}

\begin{thebibliography}{99}
\bibitem{c1} https://discordapp.com/
\bibitem{c2} https://github.com/
\bibitem{c3} https://dan-in-ca.github.io/SIP/
\end{thebibliography}

\end{document}
